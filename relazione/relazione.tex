\documentclass[final, smallextended]{svjour3}
%opening
\usepackage[11pt]{extsizes}
\usepackage[top=2cm,bottom=3cm,left=2.3cm,right=2.3cm]{geometry}

\title{Relazione progetto di Ingegneria del software 2019}
\author{Edoardo Zorzi, Elia Piccoli, Marian Statache}
\date{Luglio 2019}
\usepackage{tikz}
\usepackage{pgfopts}
\usepackage{./tikz-uml}
\usepackage{babel}[italian]
\usepackage[latin1]{inputenc}
\begin{document}

\maketitle
\tableofcontents
\newpage

\section{Ingegneria e sviluppo}
\subsection{Metodologia del processo di sviluppo}
La dimensione del team di sviluppo --- tre persone --- e i diversi livelli di interesse relativi agli `ambiti' di programmazione concernenti tale progetto, espressi inizialmente dai soggetti del team, hanno portato alla decisione di suddividere in modo moderatamente netto i compiti assegnati alle diverse persone, perlomeno almeno inizialmente. 

Specificatamente, la decisione � stata quella di assegnare a Marian il compito di creare e sviluppare la maggior parte delle interfacce grafiche di tutti i componenti costituenti il sistema nel suo complesso e quello di ideare e sviluppare la GUI generale, in particolar modo di considerare i modi con cui gli utenti si aspettano di interagire con il sistema e quindi di sviluppare le relative interfacce grafiche accordatamente.
Ad Elia ed Edoardo invece � stato assegnato il compito di ideare e sviluppare la back-end, l'architettura del sistema generale e le interfacce di comunicazione e interazione dei diversi componenti; pur non avendo definito inizialmente la suddivisione specifica di questi compiti, nel corso dello sviluppo del sistema, a seguito degli iterati processi di specificazione e design, ...  
%
%\subsubsection{Ingegneria dei requisiti}
%
%\subsubsection{Design del sistema}
%
%\subsubsection{Implementazione}
%
%\subsubsection{Verifica e convalida}
%\subsection{Design e struttura del sistema}
%
%\section{Specifica del sistema}

\subsection{Casi d'uso}
\begin{tikzpicture}
\begin{umlsystem}[x=-.5, fill=red!5]{Casi d'uso}
\umlusecase[x=1.5]{Inserisce dati anagrafici del paziente}
\umlusecase[y=-1.2, x=.95]{Inserisce dati somministrazione}
\umlusecase[x=5.8, y=-2.2]{Inserisce note stato paziente}
\umlusecase[y=-3.5, x =1.3]{Visualizza dati paziente (ultime due ore)}
\umlusecase[x=3, y=-5]{Inserisce diagnosi d'ingresso}
\umlusecase[x=3, y=-6.2]{Aggiunge prescrizioni}
\umlusecase[x=2.9,  y=-7.4]{Spegne allarmi}
\umlusecase[x=.6, y=-9]{Visualizza/stampa reports pazienti}
\umlusecase[x=0, y=-10.2]{Compila lettera di dimissioni}
\end{umlsystem}

\umlactor[x=-5.5]{Infermiere}
\umlactor[x=-5.5, y=-9.7]{Primario}
\umlactor[x=-5.5, y=-5.7]{Medico}

\umlassoc{Infermiere}{usecase-1}
\umlassoc{Infermiere}{usecase-2}
\umlassoc{Infermiere}{usecase-4}

\umlassoc{Medico}{usecase-4}
\umlassoc{Medico}{usecase-5}
\umlassoc{Medico}{usecase-6}
\umlassoc{Medico}{usecase-7}

\umlassoc{Primario}{usecase-4}
\umlassoc{Primario}{usecase-8}
\umlassoc{Primario}{usecase-9}
%\umlassoc{subuser}{usecase-3}
%\umlassoc{admin}{usecase-5}
%\umlassoc{admin}{usecase-6}
\umlinherit{usecase-5}{usecase-1}
\umlVHextend{usecase-3}{usecase-2}
%\umlinclude[name=incl]{usecase-3}{usecase-4}
%
%\umlnote[x=7, y=-7]{incl-1}{note on include dependency}
\end{tikzpicture}

\end{document}
